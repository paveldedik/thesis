\chapter{Introduction}

Memory is perhaps one of the most fundamental elements of all animals, various types of biological storages evolved as a part of the body that helps animal species survive in their harsh environment. Memory, the ability to store knowledge of the world and modify behavior of individuals accordingly, is thus essential for humans and other animals.

Human memory has been studied by psychologists and neurologists for decades as it is the biological storage for our memories and experience. Memory is further very closely related to \textit{learning} and \textit{forgetting}, both of which additionally depend to some extent on memory. That is true because our memory provides the means and structure to link new knowledge faster by association and inference. Hence, human memory is a \textit{complex system} with many interconnected components (speaking abstractly) and understanding the whole underlying process is very hard. Nevertheless, it has been the interest of many researchers who came up with all kinds of theories. The theories usually describe the phenomenon by approximating the results of observations.

In computer science, exploring and modeling human memory has applications particularly in \textit{adaptive educational systems}, which are systems that provide environment for learning different domains of educational content adaptively. In adaptive educational systems, our effort is to create very accurate representation of students in order to make the system personalized, increase students' motivation and the speed of learning. This formation of new ways of education started in view of the increasing popularity of computers and the growth of the Internet. Today, most schools and other educational institutions use the Internet and various modern technologies as a valuable tool for educating students.

Mathematical models for application in adaptive practice systems are very often based on \textit{machine learning} techniques. Their target is modeling learning of individual students, adapting to their knowledge and behavior. A lot of research has also been done by psychologists who tried to model memory in laboratory settings where the students had no prior knowledge of the learned material. Our thesis is concerned almost exclusively with the students whose prior knowledge of the practiced material varies greatly. The analysis and evaluation of models is performed using the data from the on-line system \url{slepemapy.cz} for practicing geography.

The first chapter covers the background for our thesis, we summarize the most relevant aspects of human memory and forgetting related to our research, we also describe characteristics of adaptive educational systems as well as some mathematical models used participially for student modeling. In the second chapter, we propose models and their modifications that focus on timing information, we also describe the numerous methods and metrics used for parameter estimation. In the last chapter chapter, we evaluate and compare the proposed models on several data sets containing different types of places (countries, mountains, rivers, etc.), we analyze the stability of each model's predictions and prepare perform further analysis.
