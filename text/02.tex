\chapter{Background}

In this chapter we present an overview of the related research with the topic of our thesis. First, we describe how the human brain can acquire a new knowledge, the process of retention of information and the retrieval of information from memory. Second, we give an overview of the techniques suitable for educational systems with the focus on adaptive learning of facts.

\section{Learning, Memory and Performance}

Human brain is a complex system with the capacity to gain, store and retrieve new information. This ability is into some extent identical to all mammals. The long-term memory of humans can be thought as either procedural or declarative. The procedural memory goes beyond our conscious awareness and makes us capable of the incremental acquisition of both the motor and cognitive skills. The declarative memory (sometime referred as explicit memory) is the conscious knowledge such as the world's countries or the English vocabulary.

The study of learning and memory can be divided into the following components:

\begin{itemize}
  \item Learning
  \item Memory
  \item Performance
\end{itemize}

Learning is the process of encoding information. For example the encoding of the location of Portugal can be seen as learning the shape of the country and its neighbor Spain. Memory is the biological storage that retains the information -- in our example the location of Portugal. Performance is the speed and precision of recall (retrieval of an item from memory).

\subsection{Learning}

\subsection{Memory}
