\chapter{Background}

In this chapter we present an overview of the related research with the topic of our thesis. Firstly, we describe how the human brain can acquire a new knowledge, the process of retention of information and the retrieval of information from memory. Secondly, we give an overview of the techniques suitable for educational systems with the focus on adaptive learning of facts.

\section{Learning, Memory and Performance}

\todo{This whole section needs to be covered more thoroughly, it should contain information from various resources. The reader has to be more interested in the topic at the end of this chapter.}

Human brain is a complex system with the capacity to gain, store and retrieve new information. This ability is into some extent identical to all mammals. The long-term memory of humans can be thought as either procedural or declarative. The procedural memory goes beyond our conscious awareness and makes us capable of the incremental acquisition of both the motor and cognitive skills. The declarative memory (sometime referred as explicit memory) is the conscious knowledge such as the world's countries or the English vocabulary.

The study of learning and memory can be divided into the following components:

\begin{itemize}
  \item Learning
  \item Memory
  \item Performance
\end{itemize}

\subsection{Learning}

\todo{Figures displaying the power law of learning and forgetting might help.}

Learning is the process of encoding information. For example the encoding of the location of Portugal can be seen as learning the shape of the country,  its neighbor Spain and the surrounding ocean. The learning respects the power function.

In order to be able to model learning we have to measure memory strength. Memory strength can be measured by the acquisition of these three properties:

\begin{description}
  \item[Probability of recall] Probability of the student recalling the item.
  \item[Latency of recall] Latency of the student retrieving the item from their memory.
  \item[Savings in relearning] How much time the student has to spend in order to recall the item.
\end{description}

We can further distinguish the following levels of learning as measured by memory strength:

\begin{description}
  \item[Familiarity] The student has feeling they knew the item in the past but cannot remember anymore.
  \item[Recognition] The student recognized the item when presented multiple-choice options but couldn't remember otherwise.
  \item[Recall] The student is able to recall the item with some effort.
  \item[Automaticity] The student recalls the item instantaneously when presented. Note that the level of automaticity can be measured by the latency of recall.
\end{description}

\subsection{Memory}

Memory is the biological storage that retains the information, for example the location of Portugal. The student's memory, however, decays with the time. This memory decay is called \textit{forgetting} and similarly as learning it respects the power function.

Forgetting can be reduced by repetition. Repetition can be massed or spaced. In a massed presentation the item is revised in a short interval many times over. In contrast, a spaced presentation usually consists of revisions performed in a longer period of time with pauses between presentations. It is well known that a spaced presentation leads to a better memory strength. This phenomenon is called the \textit{spacing effect}.

\subsection{Performance}

Performance is the speed and precision of recall (retrieval of an item from memory). The student's performance can be examined by a multiple-choice test, where the correctness of answers and response time is measured.

\section{The Spacing Effect}
\label{spacing-effect}

In the ACT-R model, the memory strength $m$ of the student $s$ can be modeled by the Equiation~\ref{eq-actr}.

\begin{equation} \label{eq-actr}
  m_n(t) = \ln{\sum_{i=1}^{n} t_{i}^{-d}}
\end{equation}

The parameter $t$ is a vector of seconds which passed since each of the $n$ repetitions was performed by the student $s$. The parameter $d$ represents memory decay. The problem of this equation is that it doesn't take into account the spacing effect.

\todo{Nice plot showing what the hell this is all about.}

Philip~I.~Pavlik and John~R.~Anderson developed an extended version of the equation in which the decay is a function of the activation at the time the item was presented (see Equation~\ref{eq-pavlik-decay} and Equation~\ref{eq-pavlik-activation}).

\begin{equation} \label{eq-pavlik-decay}
  d_i = ce^{m_{i-1}} + a
\end{equation}
\begin{equation} \label{eq-pavlik-activation}
  m_n(t) = \ln{\sum_{i=1}^{n} t_{i}^{-d_i}}
\end{equation}

The parameters $c$ and $a$ affect the scale of the decay. Since $m_0 = -\infty$, the value of $d_i$ is always equal to $a$ for the first practice of the student. Additionally, when $c = 0$, the result of the equation is equivalent with the Equation~\ref{eq-actr}. Because the computation is recursive and a bit complex, we present the pseudo-code of the algorithm computing the memory activation function (see Algorithm~\ref{alg-memory-activation}).

\begin{algorithm}
  \caption{The function $\textsc{MemoryActivation}: \mathbb{N}^n \rightarrow \mathbb{R}^n$ takes the vector parameter $t$ in descending order, e.g. $[56800, 56400, 3600, 60, 0]$ (the last zero is the current practice). The result of the computation is a vector $m$ of student's memory strengths during each practice.}
  \label{alg-memory-activation}
  \begin{algorithmic}[1]
    \Function{MemoryActivation}{$t$}
      \State $n \gets size(t)$
      \State $m_0 \gets -\infty$
      \For{$i \gets 1$ \textbf{to} $n-1$}
        \State $s \gets 0$
        \For{$j \gets 1$ \textbf{to} $i$}
          \State $d_j \gets ce^{m_{j-1}} + a$
          \State $s \gets s + (t_j - t_i)^{-d_j}$
        \EndFor
        \State $m_i \gets \log(s)$
      \EndFor
      \State \Return $m$
    \EndFunction
  \end{algorithmic}
\end{algorithm}

\todo{Proof of correctness? Perhaps not necessary.}

Note that the time complexity of the function is $\mathcal{O}\left(\frac{n(n-1)}{2}\right)$.

\section{Relevant Models}
\label{relevant-models}

In this section we discuss the models relevant to our work. There are two main components that concern us. The first is the estimation of prior knowledge which can be helpful if we need better understanding of the student's acquired knowledge. The second is the estimation of the current knowledge which combines the knowledge the student already had and the knowledge they acquired. Estimation of current knowledge is significant for the better part of this thesis.

\subsection{Elo System}
\label{elo}

Elo system is a mathematical model suited for student modeling as recent research has shown. Elo rating system and its extension Glicko is otherwise a very popular approach in competitor-versus-competitor games such as chess. In our work we employ the model for the estimation of prior knowledge of students.

In the standard version of Elo system we have the student's skill $\theta_s$ and the difficulty of an item $b_i$. The Equations~\ref{eq-elo-skill},~\ref{eq-elo-difficulty} demonstrate an update of the student's skill and the item's difficulty after one answered question. The parameter $R$ is $0$ or $1$ depending on the correctness of the student's answer. The probability that a student with a given skill $\theta_s$ will answer correctly on the presented question of difficulty $b_i$ is estimated by a logistic function (see Equation~\ref{eq-elo-logistic}).

\begin{equation} \label{eq-elo-skill}
  \theta_s \gets \theta_s + K(R - P(R = 1|s,i))
\end{equation}

\begin{equation} \label{eq-elo-difficulty}
  b_i \gets b_i - K(R - P(R = 1|s,i))
\end{equation}

\begin{equation} \label{eq-elo-logistic}
  P(R = 1|s,i) = \frac{1}{1 + e^{-(\theta_s - b_i)}}
\end{equation}

The constant parameter $K$ affects the change in the estimates $\theta_s$ and $b_i$. Higher value means faster change after few questions, in contrast lower value makes the change slower. This is a problem since the number of answers varies throughout the time. It's been demonstrated that the use of an uncertainty function $\frac{a}{1 + bn}$, which considers the number of answers of students in the system, makes the predictions more stable and accurate.

\subsection{Performance Factor Analysis}
\label{pfa}

Performance Factor Analysis (PFA) is a student modeling approach based on the Learning Factor Analysis (LFA). The standard PFA equation is formulated with the incorporation of knowledge components (KCs), which may include skills, concepts or facts.

\begin{equation} \label{eq-pfa-standard}
  m(i,j \in KCs,s,f) = \sum_{j \in KCs} \beta_j + \gamma_j s_{i,j} + \delta_j f_{i,j} 
\end{equation}

\begin{equation} \label{eq-pfa-standard-p}
  P(m) = \frac{1}{1 + e^{-m}}
\end{equation}

In Equation~\ref{eq-pfa-standard}, where $m$ is a function of the student's knowledge, the parameter $\beta_j$ is the difficulty of the knowledge component $j$. The counts of current successes and failures of the student $i$ and the knowledge component $j$ are represented by the parameters $s_{i,j}$ and $f_{i,j}$, where $\gamma_j$ and $\delta_j$ define the weight of each success and failure.

The standard PFA model is defined in terms of knowledge components which is not always needed. The main disadvantage, however, is the inability to consider the order of answers. Another problem with the standard model is that it doesn't take into account the probability of guessing. Both these issues are solved in the following model:

\begin{equation} \label{eq-pfa-extended}
  m \gets \begin{cases}
            m + \gamma \cdot (1 - P(m)) & \text{\textbf{if }} \text{the answer was correct} \\
            m + \delta \cdot P(m) & \text{\textbf{if}}
          \end{cases}
\end{equation}

\begin{equation} \label{eq-pfa-standard-p}
  P(m) = \frac{1}{n} + \left(1 - \frac{1}{n}\right)\frac{1}{1 + e^{-m}}
\end{equation}

The initial value of $m$ in Equation~\ref{eq-pfa-extended} can be estimated from the Elo model which was discussed in the section~\ref{elo}, i.e. $m = \theta_s - b_i$. The parameter $n$ in the Equation~\ref{eq-pfa-standard-p} matches the number of options in multiple-choice question. Note that the probability of guess is $\frac{1}{n}$ and the probability of slip is $1 - \frac{1}{n}$.

Another variation of the original PFA model was proposed by Yue Gong. The idea of the model is based on the fact that we expect the student who answered in total of four presentations two times correctly in the last two presentations to perform better than the student who answered correctly in the first two presentations. The extended model introduces a decay factor $\xi$ that changes the behavior of the parameters $s_{i,j}$ and $f_{i,j}$ (see Equations~\ref{eq-pfa-gong-s},~\ref{eq-pfa-gong-f}).

\todo{Some examples would be nice.}

\begin{equation} \label{eq-pfa-gong-s}
  s_{i,j} = \sum_{k=1}^{n-1} y_k \cdot \xi^{n-1-k}
\end{equation}

\begin{equation} \label{eq-pfa-gong-f}
  f_{i,j} = \sum_{k=1}^{n-1} |y_k - 1| \cdot \xi^{n-1-k}
\end{equation}

The parameter $y_k$ represents the correctness of the $k$-th question. 
The problem of this model is that it cannot be easily adjusted so that it includes the probability of guessing, particularly in cases where the practices were presented in the form of multiple-choice questions with varied number of options.
