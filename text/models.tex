\chapter{Models Based on Timing Information}

The models presented in chapter~\ref{relevant-models} seem to work well in the context of adaptive systems. But could the performance be further improved by taking into account the timing information of students' answers (i.e. the response times or the breaks between practices)? As was discussed in chapter~\ref{spacing-effect}, the spacing effect is observable in environments where the students have no prior knowledge. We are interested primarily in domains where the prior knowledge varies widely between students.

\section{PFA and Timing}

In this chapter we discuss several models which aim at modeling students' memory with the usage of timing information of answers. There are several several issues to consider beforehand:

\begin{itemize}
  \item Students have different skills and thus the rate of retention loss varies.
  \item The difficulty of items varies, some items are forgotten faster and some slower.
  \item The prior knowledge of each student is different. If the student already has the practiced item in their long term memory, the process of forgetting is much slower.
\end{itemize}

\subsection{PFA with Forgetting}

One possibility how to incorporate timing information into the PFA model is by changing the memory activation in prediction. The times of previous attempts are passed to a \textit{time effect function} which may increase the probability of recall (see Equation~\ref{eq-pfa-standard-time-p}).

\begin{equation} \label{eq-pfa-standard-time-p}
  P(m) = \frac{1}{1 + e^{-(m + f(t))}}
\end{equation}

Since this is an extension of the standard PFA model, it is possible to integrate the estimation of prior knowledge as well as the probability of guessing into the model.

\subsection{PFA Gong with Forgetting}

Another way of dealing with timing between attempts is by changing the decay factor $\xi$ of the model presented in chapter~\ref{pfa}. The model takes into account the order of questions, yet doesn't consider timing between student's practices of an item. This problem can be resolved by replacing the parameter $\xi$ with a time effect function.

\begin{equation} \label{eq-pfa-gong-time-s}
  s_{i,j} = \sum_{k=1}^{n-1} y_k \cdot f(t_k)
\end{equation}

\begin{equation} \label{eq-pfa-gong-time-f}
  f_{i,j} = \sum_{k=1}^{n-1} |y_k - 1| \cdot f(t_k)
\end{equation}

Equations~\ref{eq-pfa-gong-time-s},~\ref{eq-pfa-gong-time-f} show the incorporation of the time effect function $f$ in the model. The parameter $t_k$ represents the number of seconds that passed between the $k$-th practice and the most recent one. The weight of successes and failures is thus dependent on the ages of the prior practices.

As was discussed in the chapter~\ref{pfa}, the problem arises with multiple-choice questions. Another difficulty is the choice of a time effect function that fits the data well. The function should theoretically represent the rate at witch the effect of learning decays with the passage of time.
