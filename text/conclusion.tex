\chapter{Conclusion}

In the first part of the thesis, we summarized the key aspects of human memory---learning and forgetting. We discussed applications and properties of adaptive educational systems, basics of student modeling and the most relevant models convenient in the context of our thesis. We described the Elo model successfully used for estimation of prior knowledge of students and mainly the Performance Factor Analysis (PFA) used for estimation of the current knowledge students acquire while interacting with an adaptive system. We also studied some extensions of the PFA model and the models used in the ACT-R modeling system.

The primary objective of our thesis was to explore relevant models and design their extensions which take into account the key aspects of human memory (mainly forgetting). We described several models based on previous extensions and introduced a \textit{time effect function} which penalizes the age of student's past attempts. Next, we summarized machine learning techniques used for parameter estimation and quantification of model's performance. The evaluation of models was performed using data from the adaptive practice system Outline Maps. Finally, we analyzed students' response times for the possibility that it may indicate level of student's knowledge, and the values of model's parameters depending on the domain of the used data set---the type of place, or the purpose of practice.

Our experiments demonstrated that both student models which take into account ages of past trials results in a better performance. Even though the knowledge of a country leads to a faster response time, our model showed minor improvements when the duration of past response times was considered. The analysis of the model with a decay factor shows that penalizing old trials based on order of answers leads in most cases to a better performance than the penalty of trials based on timing information.

We suggested the usage of the extended PFA model with a \textit{staircase function} in production, this method is computationally very efficient and does not require periodic estimations of parameters. Other models do not perform so well especially in environments where the test of student's knowledge involves answer to a multiple-choice question. Future research may focus on analysis of models in other adaptive practice systems (e.g. \url{practiceanatomy.com}), where more complex patterns of human memory (e.g. the \textit{spacing effect}) might be more apparent.
